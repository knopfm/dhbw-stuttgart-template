%!TEX root = ../main.tex

%% Dokumentenklasse
\documentclass[%
    paper=a4,           % DIN A4
    12pt,               % Schriftgröße 12
    parskip=full,       % eine Zeile Absatzabstand
    oneside             % einseitig
    listof=totoc,		% alle Verzeichnisse in ToC einbinden
    bibliography=totoc,
    toc=listof,
    toc=chapterentrydotfill % ToC Punkte
]{scrreprt}             % KOMA Skript Report

%% Standard-Pakete
\usepackage{xstring}		            % für Stringvergleich
\usepackage[utf8]{inputenc}         	% Quelldateicodierung
\usepackage[T1]{fontenc}	            % Fontmap-Kodierung, diese wird von der pdflatex-Engine benötigt
\usepackage[english, ngerman]{babel}	% Sprache

%% ifDocType Befehlsdefinition
\newcommand{\ifDocType}[3]{%
	\IfStrEq{\documentType}{#1}{#2}{#3}%
}

%% Seiteneinstellungen
% Ränder
\usepackage[
    left=2.5cm,
    right=2.5cm,
    bottom=4cm,
    top=4cm
]{geometry}

\usepackage[onehalfspacing]{setspace}   % Zeilenabstand

% Schriftart
\usepackage{helvet}     % Arial like
\renewcommand{\familydefault}{\sfdefault}

% Kopf- und Fußzeile
\usepackage[headsepline=1pt, footsepline=1pt]{scrlayer-scrpage}
\renewcommand*\chapterpagestyle{scrheadings}
\pagestyle{scrheadings}
\ifDocType{T3\_3100}{%
    % kein Logo bei T3100
}{%
    \lohead{\includegraphics[height=1.5cm]{resources/images/logo-company.png}}
}
\chead{}
\rohead{\includegraphics[height=1.6cm]{resources/images/logo-dhbw.png}}
\newcommand{\documentTypePhrase}{%
	\IfStrEqCase{\documentType}{%
		{T3\_1000}{Projektarbeit T1000}%
		{T3\_2000}{Projektarbeit T2000}%
		{T3\_2001}{Projektarbeit T2000, Teil 1}%
		{T3\_2002}{Projektarbeit T2000, Teil 2}%
		{T3\_3000}{Projektarbeit T3000}%
		{T3\_3100}{Studienarbeit}
		{T3\_3200}{Studienarbeit}%
		{T3\_3300}{Bachelorarbeit}%
	}
}
\ifoot{\documentTypePhrase}
\cfoot{\documentAuthor}
\ofoot{Seite | \thepage}
\setlength{\marginparwidth}{2cm}

\usepackage{scrhack}    % KOMA Skript Fehlermeldung

%% Sonstige Pakete
\usepackage{todonotes}              % Todos in GROßBUCHSTABEN
\usepackage{graphicx}               % Bilder
\graphicspath{{resources/images/}}  % Standard Bilderpfad
\usepackage{subcaption}             % Mehrere Bilder/Tabllen in einer Figure
\usepackage{tikz}                   % für komplexe Bilderstellung
\usepackage{tabularx}               % Tabellen
\usepackage{pdfpages}               % PDF Dateien / Seiten
\usepackage{float}                  % Floating Darstellung
\usepackage{xcolor}                 % Farben
\usepackage{acronym}                % Abkürzungen, nur die Verwendeten: \usepackage[printonlyused]{acronym}
\usepackage{amsfonts}               % Mathematische Schriftart der American Mathematical Society
\usepackage{amsmath}                % Mathematische Schriftzeichen der American Mathematical Society
\usepackage{fancyvrb}               % [für Quelltext]
\usepackage{xurl}                   % URL Umbruch
\usepackage{pdflscape}              % Querformat
\usepackage{enumitem}               % Aufzählungsstile

%% Bibliographie
\usepackage[
	backend=biber,			% Recommended. Alternative: biblatex
	bibwarn=true,
	bibencoding=utf8,		% Wenn die .bib-Datei mit utf8 kodiert ist, sonst ascii
	sortlocale=de_DE,
	style=\quoteStyle,
]{biblatex}
\loadbibresources			% Lädt die Resourcen aus der config Datei
\newcommand{\mkbibnodate}{o\adddot J\adddot}        % o. J.
\AtEveryBibitem{\iffieldundef{year}{\restorefield{year}{\mkbibnodate}}{}}
\usepackage{csquotes}               % Zitierung mit Sprache verknüpfen

%% Querverweise und Links
\definecolor{LinkColor}{HTML}{00007A} % Farbe
\usepackage[%
    pdftitle={\documentTitle},
    pdfauthor={\documentAuthor},
    pdfsubject={\documentType},
    pdfcreator={pdflatex, LaTeX with KOMA-Script},
    pdfpagemode=UseOutlines,
    pdfdisplaydoctitle=true,
    pdflang={de},
    colorlinks = false,
	linkcolor=LinkColor,
	citecolor=LinkColor,
	filecolor=LinkColor,
	menucolor=LinkColor,
	urlcolor=LinkColor,
	linktocpage=true,
	bookmarksnumbered=true
]{hyperref}

%% Quelltext
\usepackage{listings}

\definecolor{comment}{HTML}{0A943F}
\definecolor{keyword}{HTML}{184FDB}
\definecolor{background}{HTML}{F2F2F2}
\definecolor{string}{HTML}{DB9418}
\lstset{%
    backgroundcolor=\color{background},
    basicstyle=\ttfamily\footnotesize,
    breakatwhitespace=false,
    breakautoindent=true,
    breaklines=true,
    captionpos=b,
    commentstyle=\color{comment},
    keepspaces=true,
    keywordstyle=\color{keyword},
    morekeywords={var}
    numbers=left,
    numbersep=1em,
    numberstyle=\tiny,
    postbreak=\space,
    showspaces=false,
    showstringspaces=false,
    showtabs=false,
    stepnumber=1,
    stringstyle=\color{string},
    tabsize=2,
}
\lstloadlanguages{PHP,python,java,C,C++,bash}
\renewcommand\lstlistingname{Skript}
\renewcommand\lstlistlistingname{Skriptverzeichnis}
\def\lstlistingautorefname{Skript}

%% Chapter Style
\definecolor{chapterBlue}{RGB}{47,84,150}
\addtokomafont{chapter}{\color{chapterBlue}}
\addtokomafont{section}{\color{chapterBlue}}
\addtokomafont{subsection}{\color{chapterBlue}}
\addtokomafont{subsubsection}{\color{chapterBlue}}
\RedeclareSectionCommand[beforeskip=0pt,afterindent=false,afterskip=1em]{chapter}
\RedeclareSectionCommand[beforeskip=0pt,afterindent=false,afterskip=1em]{section}
\RedeclareSectionCommand[beforeskip=0pt,afterindent=false,afterskip=1em]{subsection}
\RedeclareSectionCommand[beforeskip=0pt,afterindent=false,afterskip=1em]{subsubsection}

%% Caption Style (Bsp.: Bilderunterschrift)
\addtokomafont{caption}{\small}

%% Cover Einstellungen
\title{\documentTitle}
\author{\documentAuthor}
\date{\releaseDate}
