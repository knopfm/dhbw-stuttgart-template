%!TEX root = ./main.tex

\input{content/01_Problemstellung}
\input{content/02_Grundlagen}
\input{content/03_Hauptteil}
\input{content/04_Ausblick}

%% Beispiele, können problemlos entfernt werden!
\chapter{Beispiele}
Beispiele, können problemlos entfernt werden!
\section{Literatur}
Beispiel Text \cite[S. 10]{testBuch1}
Beispiel Homepage\cite{urlId} \\

\section{Bilder}
\begin{figure}[H]
	\centering
	\includegraphics[width=0.3\linewidth]{resources/images/logo-dhbw}
	\caption{DHBW Logo}
	\label{fig:logo-dhbw}
\end{figure}

\section{Fußnote und Abkürzung}
Fußnote\footnote{Fußnote}, \ac{DHBW}

\section{Tabelle}
\begin{table}[H]
	\centering
	\caption{Tabellenbeispiel}
	\label{tab:example}
	\begin{tabular}{|l|c|r|}
		\hline
		Spalte 1 & Spalte 2 & Spalte 3 \\
		\hline
		Zeile  &  &  \\
		\hline
		& Zeile &  \\
		\hline
		&  & Zeile \\
		\hline
		\multicolumn{2}{|r|}{Verbunden}	& nicht Verbunden \\
		\hline
	\end{tabular}
\end{table}

\section{Skript}
\lstinline[language=c]|printf("Inline Code");|
\lstinputlisting[language=python,caption={Beispiel python Skript},captionpos=t,label=scr:exanple]{resources/example-script.py}
